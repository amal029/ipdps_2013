\documentclass[10pt]{article}

\begin{document}


\section{Abstract}
\label{sec:abstract-2}

\section{Introduction and problem description}
\label{sec:intr-probl-descr}

This section should motivate the problem that we are facing or will face
with advancement of computing technology.

\section{Background}
\label{sec:background}

This should give the theoretical background of the whole process. For
example, describe the application graphs formally. Introduce the
mathematical notation behind the cost functions and the graph
theory. Introduce what we mean by throughput (formally).

\section{Modeling heterogeneity of the application graphs}
\label{sec:model-heter-appl}

Describe the kind of heterogeneity that exists in the application
graph. Nodes requiring vector instructions. Nodes that are stores. How
the communication is modeled, etc.

\section{Modeling heterogeneity of the execution platform}
\label{sec:model-heter-exec}

This is the main part of the paper: Describe how the platform is
modeled. Why a dendrogram? What is the intuition behind partitioning in
steps rather than partitioning in say one go? Need to describe the
discussions we had on board before (why go bottom up for platform graph,
but go top-down for application graph).

\section{Experimental results}
\label{sec:experimental-results}

We divide this into multiple sub-sections: (1) Give a brief background
of the experimental setup. This should include the description of cross
entropy. (2) The other sections should be partitioned into different
parts within the table.


\section{Conclusions and future work}
\label{sec:concl-future-work}

We have none

\end{document}
