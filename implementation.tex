\subsection{Implementation}
\label{sec:imple}

In order to implement our heuristic we use METIS \cite{} graph
partitioning tool.

bullshit about metis.

The resource graph is generated by assuming the capabilities lie in a
fixed range. Using this we define a range from which we assign the
respective PE's capabilities. In generating them synthetically we avoid
being biased by a single architecture and we can evaluate our system
for different characteristics. The interconnect is considered to be two
dimensional mesh with varying bandwidths. Thus, making both the
computaions and communication be truly hetrogeneous.

Once we have the resource graph, we perform the clustering as describe
in section \ref{sec:gener-reso-graph}. We use metis to do the
clustering of the nodes based on communication volume min cut and load
balance constraints. The entire process is described below:

\begin{itemize}

\item The generated resource is graph is represented in the Metis
graph format. We represent the PEs capabilities as constraints of the
nodes and the link's bandiwdths as communication volume of the edges.

\item We then construct our clustered structre by halving the nodes at
each level. Metis partitions the graph by load balancing the
constraints and min cut communication. So the number of partitions
that it might provide might be less than that being requested.

\item When constructing the new virtual node, the constraints of the
previous are aggregated and the communication is calculated by summing
the minimum bandwidths of links connecting the nodes present in the
different partitions.

\item We repeat the clustering for each level, until we reach a stage
in which all the PEs form a single partition.

\end{itemize}
