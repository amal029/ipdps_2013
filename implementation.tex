\section{Implementation}
\label{sec:imple}

We use the Metis~\cite{gkar95} graph partitioning library to implement
our partitioning algorithm. We are not tied to Metis and any other graph
partitioner such as Zoltan~\cite{kdev09} or Scotch~\cite{cche08} can be
used for implementing our algorithm. Herein, we describe how we used
Metis to implement the graph partitioning.

% Metis is a graph partitioner, which implements K-way and recursive
% graph partitioning. The weights on the graph nodes are represented as
% constraints.
Each graph node can have multiple-node weights, representing different
criteria. The edges between nodes can be weighted themselves, but unlike
nodes, edges can only be decorated with a single weight. Moreover, the
unique point about Metis is that one can use a concept called
``tp-weights'', which give weightage, to different node constraints when
performing load-balancing during K-way or recursive graph
partitioning. These are the only features of Metis that we required when
implementing our algorithm. % Metis provides a number of other features,
% which the readers can lookup in~\cite{gkar95}.

\subsection{Clustering the resource-graph : building the dendrogram}
\label{sec:clust-reso-graph}

First we describe the formation of the resource cluster graph shown in
Figure~\ref{fig:res} and its equivalent dendrogram representation in
Figure~\ref{fig:7}.

\begin{itemize}

% \item The filter-graph in Figure~\ref{fig:1}, is represented as an
%   undirected graph in the Metis file format.

\item The resource graph is represented in the Metis graph format. We
  represent the PEs capabilities as constraints of the nodes and the
  link's bandwidths as communication weight on the edges.

\item We then construct our clustered structure (Figure~\ref{fig:res})
  by asking for a 2-way partition at each level of the $log_2|V_r|$
  height. Metis partitions the graph by load balancing the constraints
  and performing a minimum edge cut. The number of partitions that Metis
  provides is bounded by $|V_r|/2$. Metis might provide fewer than
  $|V_r|/2$ partitions, if they lead to a better load-balance. In such a
  case, we continue with asking a 2-way partition at further levels of
  the hierarchy, but the height of the resultant hierarchical cluster is
  less than $log_2|V_r|$. Effective computation and communication costs
  are calculated at each stage using the formulation described in
  Section~\ref{sec:clustering-topology}.

\end{itemize}

\subsection{Partitioning the filter-graph}
\label{sec:part-filter-graph}

In partitioning the filter graph, we need to balance the constraints on to
the available partitions. Metis offers the ability to load balance
multiple constraints on to different partitions based on the metric
\mbox{`tp-weight'}. We calculate the ratios between the capabilities of
different partitions and represent them as this metric in order to load
balance on to the available partitions. The steps followed are as
follows:

\begin{itemize}

\item We start at the top most level (level 3, Figure~\ref{fig:res} or
  Figure~\ref{fig:7}) in the resource-graph hierarchy, and request for
  the number of partitions equal to that of the child nodes at the next
  level, usually a 2-way partition. For example, all the filter graph
  nodes are first allocate to \texttt{A1} in Figure~\ref{fig:7}, we then
  perform a 2-way partition of this graph onto nodes \texttt{A2} and
  \texttt{B2}. All the filter graph vertices allocated to \texttt{A2} are
  then again re-partitioned onto the resource nodes \texttt{A1} and
  \texttt{C1}, same for \texttt{B2} and so on and so forth.

\item The capabilities of the resource graph are represented as ratios
  for each of the constraint as the `tp-weight' metric when partitioning
  the filter graph using Metis. For example, consider the partitioning
  onto the level 1 resource graph in Figure~\ref{fig:7}. There are 4
  clustered nodes. We generate two sets of `tp-weights', with 2
  elements each, one for each of the node sets clustered to form the
  parent level (level-2, Figure~\ref{fig:res} and
  Figure~\ref{fig:7}). Each element in the set represents the MIPS count
  and the vector capacity of the underlying clustered node. Let us
  consider node \texttt{A2}, which is effectively the clustering of
  nodes \texttt{C1} and \texttt{A1}. Thus, the MIPS capacity element, of
  the `tp-weight' set for weight set for the nodes \texttt{C1} and
  \texttt{A1} is calculated as: {$\{C1^{MIPS}_{tpw} = R^{C1}_0/R^{C1}_0
    + R^{A1}_0, A1^{MIPS}_{tpw} = R^{A1}_0/R^{C1}_0 + R^{A1}_0\}$}. Same
  is done to calculate the vector length capacity ratios. Since, the
  other two nodes \texttt{B1} and \texttt{D1} are clustered to form the
  second parent in level 2, \texttt{B2}, we end up with two sets of
  `tp-weights'. The MIPS and vector length ratios are shown in
  Figure~\ref{fig:7}. These ratios make sure that the filter nodes are
  allocated to the correct resource graph node. An example allocation of
  the heaviest filter node (statement \texttt{3}) from Figure~\ref{fig:1},
  is shown in Figure~\ref{fig:7}.

\item As we go down on to each level we partition the filter graph into
  smaller fine grained parts until we reach the lowest level. Note that
  the `tp-weight' sets are calculated for every level in the
  resource-graph hierarchy.

\item Finally, the entire filter graph ends up being partitioned onto the
  most appropriate PE in the resource-graph.

\end{itemize}


% \subsection{Load Balancing Multiple Constraints}

% Metis load balances constraints by minimizing the imbalance with that of
% an ideal value. It tries to acheive this by modifying the partitioning
% such that each consrtaint lies within certain imbalance from the ideal
% value. Since reaching an `optimal' solution is difficult (or in some
% cases impossible), it matches each of the constraint in
% \underline{order} until it finds a solution where all the constraints
% are matched within a certain tolerance level. Unfortunately, this causes
% the contraint to be matched with more and more imbalance as we move
% further, thereby giving priority to the first node constraint and then
% to the second and so on and so forth.

% In order to handle this, we chose an addition to our heuristic that
% checks whether we achieve the least imbalance in compute power in our
% partitions. This is done by clustering the nodes at each level based
% on all the possible permutations of the resource graph. After each
% clustering for a given permutation the imbalance in compute power is
% calculated as,

% \begin{equation}
% \begin{array}{c}
% desired_compute_power = total_compute_power / no_of_nodes;\\

% for each partition, for 1->n\\
% cp_n = sum( ( compute power of node ( R * R .. Rn ) ) -
% desired_compute_power )\\

% chosen partition = min( cp_n  )\\
% \end{array}
% \end{equation}

% We then pick the partition that was yielded with the minimum deviation
% in the compute power.

% \subsection{The post pass}
% \label{sec:post-pass}




%%% Local Variables:
%%% mode: latex
%%% TeX-master: "bare_conf"
%%% End:
