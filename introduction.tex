\section{Introduction and motivating example}
\label{sec:intr-motiv-example}

Today's HPC clusters consists of a large number of heterogeneous
processing elements such as CPUs, GPUs, DSPs, FPGAs, etc. Given an
application, how does one determine if there are portions of the
application that should be mapped to a GPU, others to a CPU, and still
others to an FPGA, and so on? Further, given a heterogenous collection
of GPUs (i.e., different GPUs may have different vector lengths) and a
heteorogeneous collection of CPUs (i.e., different CPUs may have
different number of cores), how do you further decide which portions of
the GPU should be allocated to a given GPU and not to another GPU? Now
let us add the final complexity: how do you do the above allocation
automatically? How do you automatically map the parallelism available in
an application to the right computation engine from among a
heterogeneous suite of engines?  This paper offers one such
approach. What if you do not have the architecture figured out yet but
you have the application? In this case, one can use our approach to
determine the type of architecture best suited to map the parallelism
available in a given application.  Our proposed framework can be used by
topology designers to quickly carry out a design space exploration to
determine the type of underlying topology best suited for a given
application.  Moreover, compiler writers can also use our framework when
vectorizing and partitioning large vector units onto parallel GPU/CPU
units.
% \textbf{AVINASH - PLS CHECK 
% THE LAST CLAIM. DO YOU DO THAT IN THIS PAPER?}

Consider the code snippet in Figure~\ref{fig:2} that carries out the
main stencil computation using the Jacobi algorithm. Jacobi is an
important stencil computation, which is used for solving large systems
of linear equations, especially for heat transfer problems and numerical
fluid mechanics. We have chosen this as our motivating example because
there have been attempts to parallelize Jacobi using MPI~\cite{jacobi1}
and CUDA~\cite{jacobi2} with success, making it an important problem to
solve on a heterogeneous compute cluster mixing MPI and CUDA programming
techniques and with the potential of making it faster still. The CUDA
programming techniques exploit the \textit{Single Instruction Multiple
  Data} (SIMD) potential in the Jacobi algorithm~\cite{jacobi2} by
modeling parallelism as vector computations suitable for a GPU. The MPI
approach on the other hand exploits \textit{Multiple Instruction
  Multiple Data} (MIMD) potential by modeling the parallelism across
different CPU nodes. Both techniques result in 3-4 times speedup
compared to single CPU implementations. The challenge when
\textit{re-designing} and \textit{tuning} such parallel applications to
exploit vector units \textit{or} MPI alone is well
documented~\cite{jacobi1,jacobi2}. The complexity of re-designing for a
mixture of two grows exponentially. The growth in complexity of design
space is due to a number of factors, some of which we enumerate below:


\begin{scriptsize}
  \begin{figure}[h!]
    \centering
\begin{verbatim}
 //Task and data-parallel
 for (int i=0;i<M; ++i){
  for (int j=0;j<N; ++j){
   1: A[i][j] = (i*j+2.0+2.0/N)
   2: B[i][j] = (i*j+3.0+3.0/N)
  }
 }
 for (int k=0;k<TSTEPS;++k){
  for (int i=1; i<M; ++i)
   for (int j=1; j<N; ++j)
    3: B[i][j] = 0.2*(A[i][j]+A[i][j-1]
              +A[i][j+1]+A[i-1][j])

  //Data-parallel
  for (int i=1; i<M; ++i)
   for (int j=1; j<N; ++j)
    4: A[i][j] = B[i][j]
 }
\end{verbatim}
    \caption{Example 2-dimensional Jacobi application}
    \label{fig:2}
  \end{figure}
\end{scriptsize}



\begin{itemize}

\item The vector lengths of the underlying processing elements
  differ. Intel processors have 256 bit vector instructions, while the
  GPU range varies.

  % For example, a GPU warp (the smallest vector size) varies from 16-32
  % scalar instructions. Moreover, these warps can be arranged in
  % different block sizes, which results in different effective vector
  % lengths. On the CPU the vector lengths differ themselves, but the
  % variation is much smaller. For example, the new Intel AVX units have
  % a
  % vector length of 256 bits, while the SSE units have a vector length
  % of
  % 128 bits.

\item The actual data-type results in different utilization of vector
  units. For example, a \texttt{double} type requires twice as many
  vector registers to carry out processing as compared to an
  \texttt{int} type.

\item The size of the vector length and the number of vector units
  required needs to be determined: simply dividing the data-parallel
  vector units onto the largest available vector processing elements
  does not necessarily result in good application throughput or
  latency. In an ideal scenario for very large vector computations, the
  vector units can be utilized completely and the rest of the
  data-parallelism can be exploited in parallel on a CPU unit
  iteratively in a loop.

\item The bottleneck of the communication fabric plays an important role
  in the partitioning problem. Note that in a heterogeneous compute
  cluster the communication latencies and bandwidths themselves vary.

\item Allocation of data-stores being utilized by the different
  processing elements needs to be handled.

\item Applications written in different ways result in different
  parallelism potential.

\item Finally, the scheduling problem is known to be
  NP-hard~\cite{vsar89}. Thus, we need a good heuristic solution, which
  finishes quickly and gives good results.

\end{itemize}

There are a number of other applications where parallelism plays an
important
role. % (\textbf{THIS STATEMENT IS TOO OBVIOUS TO BE MADE A POINT
  % OF.  UNLESS YOU ARE SAYING THAT THESE APPS HAVE POTENTIAL FOR
  % EXPLOITING HETEROGENOUS RESOURCES. ARE YOU SAYING THAT?})
For example, binomial option pricing~\cite{ssol10,hpra10},
k-means~\cite{jzha11}, Gauss-Seidel stencil computations~\cite{hcou09},
etc, are well suited to be optimized across heterogeneous HPC
architectures. Exploiting both data and task parallelism is essential in
the general case.

The rest of the paper is arranged as follows:
Section~\ref{sec:related-work}, describes the related work and positions
our approach in comparison to this
work. Section~\ref{sec:preliminaries}, formalizes the problem
statement and defined the objective function. Next, in
Section~\ref{sec:our-framework}, we provide a detailed description of
our framework. In Section~\ref{sec:imple}, we then describe the
implementation of our approach. Section~\ref{sec:experiments-results},
give the quantitative comparisons of our approach against a number of
other approaches. Finally, we conclude in Section~\ref{sec:conclusion}.

% \section{Introduction and motivating example}
% \label{sec:intr-motiv-example}

% Today's HPC clusters consists of a large number of heterogeneous
% processing elements such as CPUs, GPUs, DSPs, FPGAs, etc. Given an
% application exhibiting potential for parallelism the question remains:
% how does one determine the type of architecture best suited to extract
% this parallelism? Or given an architecture, how does one determine how
% to exploit the potential parallelism in the applications.

% \begin{scriptsize}
%   \begin{figure}[h!]
%     \centering
% \begin{verbatim}
%  //Task and data-parallel
%  for (int i=0;i<M; ++i){
%   for (int j=0;j<N; ++j){
%    1: A[i][j] = ((((i)*((j)+2.0))+2.0)/(N))
%    2: B[i][j] = ((((i)*((j)+3.0))+3.0)/(N))
%   }
%  }
%  for (int k=0;k<TSTEPS;++k){
%   for (int i=1; i<M; ++i)
%    for (int j=1; j<N; ++j)
%     3: B[i][j] = 0.2*(A[i][j]+A[i][j-1]
%               +A[i][j+1]+A[i-1][j])

%   //Data-parallel
%   for (int i=1; i<M; ++i)
%    for (int j=1; j<N; ++j)
%     4: A[i][j] = B[i][j]
%  }
% \end{verbatim}
%     \caption{Example 2-dimensional Jacobi application}
%     \label{fig:2}
%   \end{figure}
% \end{scriptsize}

% Consider the code snippet in Figure~\ref{fig:2} that carries out the
% main stencil computation using the Jacobi algorithm. Jacobi is an
% important stencil computation, which is used for solving large systems
% of linear equations, especially for heat transfer problems and numerical
% fluid mechanics. We have chosen this as our motivating example, because
% there have been attempts to parallelize Jacobi using MPI~\cite{jacobi1}
% and CUDA~\cite{jacobi2} with success, making it an important problem to
% solve on a heterogeneous compute cluster mixing MPI and CUDA programming
% techniques and with the potential of making it faster still. The CUDA
% programming techniques exploit the \textit{Single Instruction Multiple
%   Data} (SIMD) potential in the Jacobi algorithm~\cite{jacobi2} by
% modeling parallelism as vector computations suitable for a GPU. The MPI
% approach on the other hand exploits \textit{Multiple Instruction
%   Multiple Data} (MIMD) potential by modeling the parallelism across
% different CPU nodes. Both techniques result in 3-4 times speedup
% compared to single CPU implementations. The challenge when
% \textit{re-designing} and \textit{tuning} such parallel applications to
% exploit vector units \textit{or} MPI alone is well
% documented~\cite{jacobi1,jacobi2}. The complexity of re-designing for a
% mixture of two grows exponentially. The growth in complexity of design
% space is due to a number of factors, some of which we enumerate below:

% \begin{itemize}

% \item The vector lengths of the underlying processing elements
%   differ. Intel processors have 256 bit vector instructions, while the
%   GPU range varies.

%   % For example, a GPU warp (the smallest vector size) varies from 16-32
%   % scalar instructions. Moreover, these warps can be arranged in
%   % different block sizes, which results in different effective vector
%   % lengths. On the CPU the vector lengths differ themselves, but the
%   % variation is much smaller. For example, the new Intel AVX units have
%   % a
%   % vector length of 256 bits, while the SSE units have a vector length
%   % of
%   % 128 bits.

% \item The actual data-type results in different utilization of vector
%   units. For example, a \texttt{double} type requires twice as many
%   vector registers to carry out processing as compared to an
%   \texttt{int} type.

% \item The size of the vector length and the number of vector units
%   required needs to be determined: simply dividing the data-parallel
%   vector units onto the largest available vector processing elements
%   does not necessarily result in good application throughput or
%   latency. In an ideal scenario for very large vector computations, the
%   vector units can be utilized completely and the rest of the
%   data-parallelism can be exploited in parallel on a CPU unit
%   iteratively in a loop.

% \item The bottleneck of the communication fabric plays an important role
%   in the partitioning problem. Note that in a heterogeneous compute
%   cluster the communication latencies and bandwidths themselves vary.

% \item Allocation of data-stores being utilized by the different
%   processing elements needs to be handled.

% \item Applications written in different ways result in different
%   parallelism potential.

% \item Finally, the scheduling problem is known to be
%   NP-hard~\cite{vsar89}. Thus, we need a good heuristic solution, which
%   finishes quickly and gives good results.

% \end{itemize}

% There are a number of other applications where parallelism plays and
% important role. For example, binomial option pricing, k-means
% calculations, Gauss-Seidel stencil computations, etc, are well suited to
% be optimized across heterogeneous HPC architectures. In general we have
% found it is much more essential to exploit data-parallelism as compared
% to task-parallelism to achieve speedups. But, exploiting both types of
% parallelism is essential in the general case.

% In this paper we propose a framework, which can be used by topology
% designers, and application writers to quickly carry out a design space
% exploration to determine the type of underlying topology best suited for
% a given application. Moreover, compiler writers can also use our
% framework when vectorizing to partition large vector units onto parallel
% GPU/CPU units.



%%% Local Variables: 
%%% mode: latex
%%% TeX-master: "bare_conf"
%%% End: 
