% -*- mode:latex; mode:flyspell -*-
\subsection{Generating the filter graph}
\label{sec:build-appl-graph}

% \begin{figure*}[t!]
%   \centering
% \begin{verbatim}
%  %scevgep = getelementptr [1000 x <1000 x float>]* %A, i64 0, i64 %3
%  %scevgep10 = bitcast <1000 x float>* %scevgep to i8*
%  %ugep = getelementptr i8* %scevgep10, i64 4
%  %scevgep11 = getelementptr [1000 x <1000 x float>]* %B, i64 0, i64 %3
%  %scevgep1112 = bitcast <1000 x float>* %scevgep11 to i8*
%  %ugep13 = getelementptr i8* %scevgep1112, i64 4
%  call void @llvm.memcpy.p0i8.p0i8.i64(i8* %ugep,i8* %ugep13,i64 3996,i32 4,i1 false)
% \end{verbatim}
%   \caption{LLVM code for assignment statement \texttt{4} from
%     Figure~\ref{fig:2}}
%   \label{fig:3}
% \end{figure*}

The filter graph is built from the application by the compiler % extracts
% filter and data parallelism form the application for partitioning the
% application onto the architecture. The compiler looks at every
% statement in the program to form the filter graph. The filter graph is
% formed
in the
following manner:

\begin{itemize}

\item The assembly instruction count for every statement is obtained
  first. Currently, we look at the LLVM (Low level virtual
  machine)~\cite{clat04} instruction count. % For example, the LLVM code
  % generated for the assignment statement \texttt{4} in
  % Figure~\ref{fig:2} is shown in Figure~\ref{fig:3}.
  The LLVM instruction count gives an approximation of the instruction
  count of the underlying hardware, while remaining independent of the
  hardware itself. The difference between $T^4_1$ and $T^0_1$ can be
  accounted for by the differing loop iteration counts and loop-collapse
  for statement 1, but not for statement 4 for illustrative purposes.

\item Every loop is fissed (split into multiple loops) thereby forming
  multiple statements. The intuition behind fissing the loops is two
  fold: (a) filter parallelism can be exploited by running independent
  loop statements separately on different machines (see
  Figure~\ref{fig:1}, where statements \texttt{1} and \texttt{2} are
  split statements) and (b) the graph partitioner would give us feedback
  on the vector size of the loop, which can then be fused back if
  allocated to the same resource.

\item The vector counts for each statement is determined using
  dependence analysis and using the polyhedral model~\cite{mgri98}. The
  largest vector size requirement is given as the second requirement
  ($T^i_1$) in the filter graph.

\item The polyhedral model is also used to find the iteration count of
  the statements and to determine the total amount of data (in bits)
  required to process by that statement. For example, statement
  \texttt{4} requires \texttt{7.8085KB} of data, while the first two
  statements \texttt{1} and \texttt{2} require 64-bits more due to the
  difference in the loop iteration count and the \texttt{double} type of
  arrays \texttt{A} and \texttt{B}. % (see Figure~\ref{fig:2})

\item The dependence edges in Figure~\ref{fig:1} are represented by solid
arcs. For example, in the Jacobi example, statements \texttt{1} and
  \texttt{2} can be carried out in parallel, while statements
  \texttt{3} and \texttt{4} have a dependence on these two statements
  and hence, cannot be split.

\end{itemize}

\paragraph{\textbf{Tiling vectors}}
\label{sec:tiling-vector-counts}

As we can see from Figure~\ref{fig:1}, three of the statements in Jacobi
example require almost a million vector instructions to be carried out
in parallel. % The vector counts can increase quite quickly for large
% examples (notice that the current 1 million vector count is just for a
% small 1000 $\times$ 1000 Jacobi matrix).
In order to properly utilize the underlying vector hardware these vector
lengths need to be split into smaller vector lengths. The resultant
vector lengths depend upon the underlying hardware. Splitting the vector
constraints is termed tiling in the compiler community. There are many
ways to tile a vector. For example, given a single processing element
with a small vector size, a vector might be tiled to fit the underlying
hardware vector size and then run in a loop (an approach taken by the
gcc compiler). If a number of processing elements with differing vector
sizes are available as in the case of HPC architectures determining the
optimal vector tiles is a challenge and can be solved with
\textit{simulated annealing} (SA)~\cite{tbra01} or \textit{genetic
  algorithm} (GA)~\cite{tbra01} heuristics. In this article we do not
solve the problem of determining the tile sizes because it is out of the scope
of this paper. Instead our framework
allows the application designers to plug and play with different vector
tiles in order to determine the tile size that suits their
architecture. We randomly generate different tile sizes for
experimentation.

An example tiling with 250K separate, 2 $\times$ 2 tiles for statement
\texttt{1} is shown in Figure~\ref{fig:1} with its resultant LLVM
code. % It is important to mention that vectors are only single
% dimensional. In Figure~\ref{fig:1}, we are type-casting a 2D matrix of
% size 2 $\times$ 2 into a single dimensional 4 element vector. Such
% vectorization is also called loop-collapsed vectorization, i.e.,
% instead of just vectorizing the inner most loop in Figure~\ref{fig:2}
% statement \texttt{1}, we are collapsing the outer and inner loop into
% a single vector instruction to improve performance. Such loop collapse
% is not necessary and can be considered a super optimization. However,
% our framework allows the application designers to explore such
% possibilities.



%%% Local Variables:
%%% mode: latex
%%% TeX-master: "bare_conf"
%%% End:
